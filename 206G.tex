\documentclass{article}
\usepackage{setup}
\begin{document}

\begin{itemize}
  \item $s$ is a strategy profile which is an assignment of strategies for each player. E.g. $s = (N,S)$ is a strategy profile for 2 players where player 1 goes north and player 2 goes south.
  \item $s_i$ is the chosen strategy (chosen action) for player $i$ given profile $s$. E.g. for the strategy profile $s = (N, S)$, $s_1 = N$, aka player 1 chooses the north strategy.
  \item $S_i$ is the set of strategies available to player $i$. E.g. $S_1 = \{N, S\}$, aka player 1 can choose either north or south. For a symmetrical game, $S_1 = S_2$.
  \item $u_i(s)$ is the utility (payoff) for player $i$ given strategy profile $s$. E.g. for the strategy profile $s = (N, S)$, $u_1(N, S) = 1$, aka player 1 receives a payoff of 1 when they choose north and player 2 chooses south.
  \item $\sigma$ is a mixed strategy profile. E.g. $\sigma = ((0.5, 0.5), (0.25, 0.75))$ is a mixed strategy profile for 2 players where player 1 chooses north with probability 0.5 and south with probability 0.5, and player 2 chooses north with probability 0.25 and south with probability 0.75.
  \item $\sigma_i$ is the mixed strategy for player $i$ given mixed profile $\sigma$. E.g. for the mixed strategy profile above, $\sigma_1 = (0.5, 0.5)$, aka player 1 chooses north with probability 0.5 and south with probability 0.5.
 \item $\sigma_i(s_i)$ is the probability that player $i$ chooses strategy $s_i$ given mixed strategy $\sigma_i$. E.g. for the mixed strategy $\sigma_1 = (0.5, 0.5)$, $\sigma_1(N) = 0.5$, aka player 1 chooses north with probability 0.5.
  \item $\Sigma_i$ is the set of mixed strategies available to player $i$. E.g. $\Sigma_1 = \{(p, 1-p)\ |\ 0 \leq p \leq 1\}$, aka player 1 can choose any probability distribution over their strategies.
  \item $u_i(\sigma)$ is the expected utility (expected payoff) for player $i$ given mixed strategy profile $\sigma$.\[u_i(\sigma) = \sum_{s\in S}\left(\prod_{j=1}^I\sigma_j(s_j)\right)u_i(s)\]
  Steps to find $u_i(\sigma)$:
    \begin{itemize}
      \item For each strategy profile $s$ in the set of all strategy profiles $S$:
      \item Calculate the probability of that strategy profile occurring by multiplying the probabilities of each player choosing their respective strategies in that profile.
      \item Multiply that probability by the utility (payoff) for player $i$ given that strategy profile.
      \item Sum these values over all strategy profiles to get the expected utility for player $i$.
    \end{itemize}
  For the mixed strategy profile $\sigma = ((0.5, 0.5), (0.25, 0.75))$, the expected utility for player $i$ is:
  \begin{align*}
      u_i(\sigma) =\ &\sigma_1(N)\sigma_2(N)u_i(N, N) + \sigma_1(N)\sigma_2(S)u_i(N, S) \\
  &+ \sigma_1(S)\sigma_2(N)u_i(S, N) + \sigma_1(S)\sigma_2(S)u_i(S, S)
  \end{align*}
  
  \item A pure strategy is discrete: either north or south for example
  \item A mixed Nash equilibrium is the best outcome for both players when they're using mixed strategies (where the strategy is a probability of going north or south rather than a discrete north or south choice).
  \item A simplex is when each component adds to 1, e.g. $x+y=1$ $\leftarrow$ 2 strategy game.
  \item $\sigma$ is a mixed profile
  \item $x=f(x)$ is a fixpoint
  \item Brouwer fixpoint theorem
  \item Kakutani fixpoint theorem
  \item Every $n$-player game has at least one Nash equilibrium.
  \item Sperner's Lemma:
    \begin{itemize}
      \item Pick an arbitrary set of $n-1$ colors (let’s say red and blue)
      \item Count red-blue edges after slicing simplex along cutlines
      \item Count exterior edges (call it $a$)
      \item Count interior edges twice (call it $b$)
      \item Total count is $a + 2b$
    \end{itemize}
\end{itemize}

\section*{Lecture 3}
\begin{itemize}
  \item A symmetric 2 player game happens when $B = A^T$, where $A$ and $B$ are the matrices of the payoff for each player.
  \item Let probability vector x be P1's mixed strategy. Let y be player 2's mixed strategy. E.g. $y=(0.5, 0.5)$.
  \item The vector of payoff for P1's pure strategies is $Ay$, where $A$ is a matrix of payoffs and $y$ is a vector of probabilities.
  \item P1 gets $x^tAy$. This ends up being a single number. P2 gets $x^tBy$.
  \item $Ay$ is a column vector of payoffs for each of p1's pure strategies.
  \item P1's best response correspondence is \[r_1(y)=\argmax_{x \ge 0 \quad\text{and}\quad ||x||_1 = 1}x^tAy\]
  \item $||x||_1=1$ means all components of $x$ add to 1.
  \item Best response can have a mix only when there are ties in the vector $Ay$.
  \item P2 best response correspondence is \[r_2(x)=\argmax_y x^tBy\]
  \item A nash equilibrium is a pair $(x^*, y^*)$ such that $x^* \in r_1(y^*)$ and $y^* \in r_2(x^*)$.
  \item For any 2 player game given by (A, B) we can define a related symmetric game $(\tilde{A}, \tilde{B})$ where \[\tilde{A} = \begin{bmatrix} 0 & A \\ B^T & 0 \end{bmatrix}\] and \[\tilde{B} = \tilde{A}^T = \begin{bmatrix} 0 & B \\ A^T & 0 \end{bmatrix}\]
  \item Claim: If $(x^*, y^*)$ is a nash equilibrium for $(\tilde{A}, \tilde{B})$, then $(\frac{x^*}{||x^*||_1}, \frac{y^*}{||y^*||_1})$ is a nash equilibrium for $(A, B)$.
  \item $Ax^*$ is a column vector of payoffs for each of p2's pure strategies.
  \item \[A = \begin{bmatrix}
    0 & -0.8 & 1.2 \\
    1 & 0 & -1 \\
    -1 & 1 & 0
  \end{bmatrix}\]
  Guess: Rock, Paper, Scissors all in support of equilibrium symmetric strategy. Call R = $x_1$, P = $x_2$, S = $x_3$. \[-0.8x_2+1.2x_3=x_1-x_3=x_2-x_1\]
  $x_3$ is equal to $1-x_2-x_1$ since they all must add to 1. Substituting in we get \[1.2-2x_2-1.2x_1=2x_1+x_2-1=x_2-x_1\]
  Solving this we get $x_1=1/3, x_2=0.377778, x_3=1-1/3-0.377778$.
  \item "In support" means it's part of the mixed strategy.
  \item You can never have a dominated strategy in a nash equilibrium.
  \item Dominated means that you could pick some other strategy such that no matter what the other player does, you always do better.
\end{itemize}

\section*{Lecture 4}
\begin{itemize}
  \item Definition 1 of a Nash Equilibrium: A strategy profile such that each player is playing a best response.
  \item Definition 2: A strategy profile and beliefs for each player about how each of the other players will play.
  \begin{itemize}
    \item Each player is maximizing payoff given those beliefs.
    \item Beliefs are correct.
  \end{itemize}
  \item Rational means ``I won't do things that I know will give me an inferior payoff.". I know my opponent(s) is rational. I know my opponent knows that I'm rational. Etc etc. This is called common knowledge of rationality.
  \item Rationalizable profiles: Profile + Beliefs for each player. \begin{itemize}
    \item Each player maximizes payoff with respect to their beliefs.
    \item Beliefs are not inconsistent with rationality being common knowledge.
  \end{itemize}
  \item Example: Sealed first price auction. Each player submits a bid without knowing the other player's bid (sealed). The highest bid wins and pays their bid (first price).
  \begin{center}
  \begin{tabular}{c|c}
    & Valuation \\
    Player 1 & 3.5 \\
    Player 2 & 1
  \end{tabular}
\end{center}
Suppose they know the valuations of each other
\begin{center}
\begin{tabular}{c|c}
& Payoff \\\hline
win & valuation - bid \\
lose & 0 \\
tie & 1/2(valuation - bid)
\end{tabular}
\end{center}

\begin{center}
  P2\\
\begin{tabular}{c|c|c|c}
& 1 & 2 & 3 \\
\hline
P1 1 & 5/4, 0 & 0, -1, & 0, -2 \\\hline
P1 2 & 3/2, 0 & 3/4, -1/2 & 0, -2 \\\hline
P1 3 & 1/2, 0 & 1/2, 0 & 1/4, -1 \\
\end{tabular}
\end{center}
\begin{itemize}
  \item For P2, strategy 1 strictly dominates strategy 3. Strategy 1 weakly dominates strategy 2 because there is a tie in the last row.
  \item You can always eliminate strictly dominated strategies. After revising the game by eliminating a dominated strategy, the other player might now have a strictly dominated strategy since it no longer considers the column/row that was eliminated.
  \item In the above example, now that we remove column 3, row 1 is strictly dominated by row 2. Furthermore, row 3 is strictly dominated by row 2.
  \item This process is called iterated deletion of strictly dominated strategies (IDSDS).
\end{itemize}
\item Another example:
\begin{center}
\begin{tabular}{c|c|c}
& P2 N & P2 S \\\hline
P1 N & 0, \circnum{0} & \circnum{10}, -10 \\\hline
P1 S & \circnum{5}, -5 & -1, \circnum{1}
\end{tabular}
\end{center}
\item If you find that a mixed nash equilibrium has a negative number, then the strategies you chose to be in support of the equilibrium were wrong.
\item There are always an odd number of Nash equilibria.
\end{itemize}

\end{document}